% we want to make a cs brief
\documentclass{lincolncsbrief}

% packages
\usepackage{blindtext}

% module name and code
\modulename{Advanced \LaTeX}
\modulecode{CMP9999M}

% assignment name and number
\assignmentname{\LaTeX\, Class File}
\assignmentnumber{1}

% the weighting (percentage) of this assignment
\weighting{60}

\begin{document}

	% make the brief header, with logo, title, assignment name/number and weightings
	\makebriefheader
		
	% learning outcomes
	\begin{learningoutcomes}
		\learningoutcome{1}{The first learning outcome.}
		\learningoutcome{2}{The second learning outcome.}
	\end{learningoutcomes}
	
	% requirements for this assignment
	\begin{requirements}
		
		\heading{Intro}
		You \textbf{must} implement the following things:
		\begin{itemize}
			\item A class file for creating assignment briefs for the University of Lincoln, School of Computer Science.
			\item A document which tests this class.
		\end{itemize}
		
		
		\heading{Class file}
		Your class file must make use of \texttt{fifo-stack}, and the font must be from \texttt{txfonts}.
				
	\end{requirements}
	
	% default information -- useful information can also be custom-made with \begin{information} ... \end{information}
	\defaultinformation
	
	% set paragraph spacing so that there is a space between paragraphs
	\setparagraphspacing
	
	% submission instructions
	\begin{submission}
		The final deadline for submission of this work is included in the School Submission dates on Blackboard. The final submission must be in the form of a single ZIP file, submitted through the Blackboard upload area for this assessment item. 
		
		% lipsum
		\blindtext
		
		\textbf{DO NOT} include this briefing document with your submission.
	\end{submission}
	
	
	
\end{document}